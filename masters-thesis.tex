% Packages (fold)
\RequirePackage{lmodern}
\documentclass[12pt, oneside, extrafontsizes]{memoir}  % TODO 12pt, twoside

\setstocksize{11in}{8.5in}
\settrimmedsize{11in}{8.5in}{*}
\settrims{0in}{0in}
\setlrmarginsandblock{25mm}{25mm}{*}
\setulmarginsandblock{25mm}{25mm}{*}
\setheadfoot{13pt}{26pt}
\setheaderspaces{*}{13pt}{*}
\checkandfixthelayout
\DoubleSpacing
\setsecnumdepth{subsubsection}
\headstyles{default}
\chapterstyle{ell}
\setsecheadstyle{\scshape\LARGE\raggedright}

\usepackage[colorlinks,bookmarksnumbered,bookmarksdepth=subsubsection,unicode=true]{hyperref}
\newsubfloat{figure}  % must follow hyperref
\hypersetup{
pdfauthor = {Matt Grenander},
pdftitle = {Learning to Balance Lead Bias in News Summarization},
pdfsubject = {Subject},
pdfkeywords = {natural language processing, automatic summarization, deep learning, machine learning},
pdfcreator = {LaTeX with the hyperref package},
pdfproducer = {},
linkcolor = [HTML]{000000},
citecolor = [HTML]{0000FF},
%urlcolor = [HTML]{\colorc}
}
\usepackage{amsmath}
\usepackage{amsfonts}
\usepackage{amssymb}
\usepackage{amsthm}
\usepackage{dsfont}
\usepackage{xspace}
\usepackage{tikz}
\usetikzlibrary{shapes,arrows}
\usepackage{standalone}
\usepackage{color,soul}
\usepackage[boxed]{algorithm2e}

% Custom commands
\newcommand{\rvs}{r.v.'s\xspace}
\newcommand{\mdp}{Markov Decision Process\xspace}
\newcommand{\mdps}{Markov Decision Processes\xspace}
\newcommand{\mrp}{Markov Reward Process\xspace}
\newcommand{\mc}{Markov Chain\xspace}

\def\given{\;\middle\vert\;}
\def\optimal{\star}
\def\transpose{\intercal}
\def\laplacian{\mathbf{\mathcal{L}}}
\def\eqdef{\overset{\underset{\mathrm{def}}{}}{=}}
\def\indicator{\mathds{1}}
\DeclareMathOperator{\expectation}{\mathbb{E}}
\DeclareMathOperator{\vol}{\text{vol}}

\newcommand{\Drandom}{$D_{\mathrm{random}}$}
\newcommand{\Dinorder}{$D_{\mathrm{inorder}}$}
\newcommand{\Dearly}{$D_{\mathrm{early}}$}
\newcommand{\Dlate}{$D_{\mathrm{late}}$}
\newcommand{\Dmedian}{$D_{\mathrm{med}}$}

\newcommand{\todo}[1]{[TODO: #1]}
\newcommand{\termidx}[1]{\index{#1}{\textbf{#1}}}

\theoremstyle{plain}
\newtheorem{thm}{Theorem}[section]
\newtheorem{lem}[thm]{Lemma}
\newtheorem{prop}[thm]{Proposition}
\newtheorem*{cor}{Corollary}

\theoremstyle{definition}
\newtheorem{defn}{Definition}[section]
\newtheorem{conj}{Conjecture}[section]
\newtheorem{exmp}{Example}[section]

\usepackage[utf8]{inputenc}
\usepackage{csquotes}
\usepackage{showidx}
\makeindex

\usepackage[backend=biber, citestyle=authoryear, bibstyle=authoryear, isbn=false, url=false, doi=false, eprint=false, natbib=false, sorting=nty, uniquename=init]{biblatex}
\addbibresource{library.bib}

\begin{document}

%%%%%%%%%%%%%%%%%%%%%%%%%%%%%%%%%%%%%%%%%%%%%%%%%%%%%
% Title page
%%%%%%%%%%%%%%%%%%%%%%%%%%%%%%%%%%%%%%%%%%%%%%%%%%%%%

% Title (fold)
\pretitle{\begin{center}\cftchapterfont\huge}
\posttitle{\end{center}}
\preauthor{\begin{center}\huge}
\postauthor{\end{center}}
\predate{\begin{center}\large}
\postdate{\end{center}}

\title{Learning to Balance Lead Bias in News Summarization}
\author{Matt Grenander}
\date{\today}
\renewcommand\maketitlehookb{
\vfill
}
\renewcommand\maketitlehookc{
\vfill
\begin{center}
{
\large
Department of Computer Science\\
McGill University, Montreal
}
\end{center}
\vspace{10mm}
}
\renewcommand\maketitlehookd{
\vspace{10mm}
\begin{center}
A thesis submitted to McGill University in partial fulfilment of the requirements of
the degree of Master of Science. \\
\copyright 2020 Matt Grenander
\end{center}
}
% Title (end)

\begin{titlingpage}
\maketitle
\end{titlingpage}

%%%%%%%%%%%%%%%%%%%%%%%%%%%%%%%%%%%%%%%%%%%%%%%%%%%%%
% Ackowledgements
%%%%%%%%%%%%%%%%%%%%%%%%%%%%%%%%%%%%%%%%%%%%%%%%%%%%%
%\clearpage
%\pagenumbering{roman}
%\renewcommand{\abstractname}{Dedication}

%%%%%%%%%%%%%%%%%%%%%%%%%%%%%%%%%%%%%%%%%%%%%%%%%%%%%
% Ackowledgements
%%%%%%%%%%%%%%%%%%%%%%%%%%%%%%%%%%%%%%%%%%%%%%%%%%%%%
\clearpage
\pagenumbering{roman}
\renewcommand{\abstractname}{Acknowledgements}
\begin{abstract}
Acknowledgements go here.
\end{abstract}

%%%%%%%%%%%%%%%%%%%%%%%%%%%%%%%%%%%%%%%%%%%%%%%%%%%%%
% Abstract
%%%%%%%%%%%%%%%%%%%%%%%%%%%%%%%%%%%%%%%%%%%%%%%%%%%%%
\clearpage
\renewcommand{\abstractname}{Abstract}
\begin{abstract}
The progression of technology and the Internet has allowed us to create and share information at an unprecedented rate.
The usefulness of summarization systems to intelligently condense text has become apparent as a method to sift through the vast amount of text we encounter each day.
Summarization systems typically follow one of two approaches.
In abstractive summarization, the system generates a summary token-by-token, whereas in the extractive approach, the system is tasked with selecting snippets of text from the source document -- typically sentences -- that best represent the text.
Extractive summarization systems have recently been shown to exhibit a bias towards selecting content from an article's lead sentences, even when these sentences are completely irrelevant to the overall text. \parencite{kedzie2018content}.
We investigate this phenomenon in detail, showing that when articles' sentence order is permuted, state-of-the-art systems cannot recover the same performance.
We then propose a solution to this problem, an auxiliary objective function which encourages the model to look beyond a document's leading sentences and properly value each sentence.
We show that adding this auxiliary objective results in significant improvements in performance, particularly in cases where the article's leading sentences constitute a poor summary.
We extend this approach to a novel summarization method that classifies documents with a strong lead vs. a weak lead, and having separate systems summarize the two cases.
We show that this approach is promising, though more work is needed in classifying articles in this way accurately.
\end{abstract}

\clearpage
\renewcommand{\abstractname}{Abrégé}
\begin{abstract}
French translation goes here.
\end{abstract}

%%%%%%%%%%%%%%%%%%%%%%%%%%%%%%%%%%%%%%%%%%%%%%%%%%%%%
% Table of content
%%%%%%%%%%%%%%%%%%%%%%%%%%%%%%%%%%%%%%%%%%%%%%%%%%%%%
\clearpage
\setcounter{tocdepth}{2}
\tableofcontents
\newpage
\listoffigures
%\listofalgorithms

%%%%%%%%%%%%%%%%%%%%%%%%%%%%%%%%%%%%%%%%%%%%%%%%%%%%%
% Introduction
%%%%%%%%%%%%%%%%%%%%%%%%%%%%%%%%%%%%%%%%%%%%%%%%%%%%%
\clearpage
\pagenumbering{arabic}
\chapter{Introduction}
\textbf{Natural Language Processing} (NLP) is an important subfield of artificial intelligence concerned with designing systems that understand and respond to human language. Complex natural language tasks often require intricate solutions, and many NLP problems remain open research areas.
Among these topics, \textbf{automatic summarization} aims to create systems able to write concise summaries of various text sources.
Automatic summarization's use cases are diverse: these systems could be used to summarize informal text such as emails or extract key points from formal literature such as medical reports or legal documents \parencite{zhang2018radsum, billsum}.
The usefulness of this task attracts substantial research interest, and forms the topic of this thesis.

Summarization systems typically follow one of two approaches. \textbf{Abstractive} summarizers generate summaries token-by-token, requiring the system to both understand the text in-depth and produce a grammatical summary. 
On the other hand, \textbf{extractive} summarizers create summaries by selecting relevant text spans -- usually sentences -- from the source document.
This thesis will focus on analyzing and improving the extractive approach, particularly in the news domain.

Extractive summarizers have historically focused on selecting the most \textit{relevant} text snippets while reducing \textit{redundancy} between selected segments. This can be accomplished in numerous ways, such as a linear combination of these two features \parencite{mmr}, or more sophisticated formulations using integer linear programming or graphical methods \parencite{mcdonald2007study, textrank}.
More recently, state-of-the-art summarizers have overwhelmingly been developed using deep learning approaches, and this thesis focuses on deficiencies within these methods.

\begin{table}[t]
    \centering
    \small
    \begin{tabular}{|p{0.94\linewidth}|}
        \hline
        \textbf{Article:} \hl{Bangladesh beat fellow World Cup quarter-finalists Pakistan by 79 runs in the first one-day international in Dhaka. Tamim Iqbal and Mushfiqur Rahim scored centuries as Bangladesh made 329 for six and Pakistan could only muster 250 in reply. Pakistan will have the chance to level the three-match series on Sunday when the second ODI takes place in Mirpur.} Bangladesh elected to bat after winning the toss but struggled to 67 for two in the 20th over after Soumya Sarkar was run out by Wahab Riaz for 20 and Mahmudullah was bowled by Rahat Ali. Tamim and Mushfiqur set about repairing Bangladesh's stuttering innings and did just that by putting on 178 runs in 21.4 overs for the third wicket, during which time Tamim notched his fifth ODI century. He continued to plunder runs with ease but went for one big shot too many against Wahab and was caught at mid-off to leave Bangladesh on 245 for three with nine overs left. His 135-ball knock of 132 included 15 fours and three sixes. Two sixes in the 43rd over took Mushfiqur into the nineties and he joined Iqbal in passing three figures by hitting Saeed Ajmal for successive fours in the 45th. Mushfiqur perished in the 48th over when he edged Wahab behind to depart for a 77-ball innings of 106 which included 13 fours and two sixes. Shakib al Hasan and Sabbir Rahman scored 30 runs in 2.2 overs before falling to Wahab (four for 59) in the final over as the hosts set pakistan 330 for victory. Azhar Ali and Sarfraz Ahmed put on 53 runs for the first wicket before the latter slog-swept Arafat Sunny to deep backward square leg to leave Pakistan one down in the 11th over. Mohammad Hafeez was then run out to leave Pakistan stuttering on 59 for two but there was some respite when Azhar notched his fifth ODI half-century in the 20th over to celebrate his first match as captain in fine style.\\ \hline
        \textbf{Reference:} Bangladesh beat fellow World Cup quarter-finalists Pakistan by 79 runs. Tamim Iqbal and Mushfiqur Rahim scored centuries for Bangladesh. Bangladesh made 329 for six and Pakistan could only muster 250 in reply. Pakistan will have the chance to level the three-match series on Sunday. \\
        \hline
    \end{tabular}
    \caption[An example in which leading sentences form a good summary.]{Leading sentences often form a strong baseline for news summarization, such as in this example. Here, the highlighted passage indicates the most closely related sentences to the reference summary. Note that the article has been truncated for conciseness.}
    \label{tab:lead_ex}
\end{table}

% Different domains carry different idiosyncrasies
Different text domains often carry idiosyncratic traits and require different summarization approaches. For example, a system summarizing emails may benefit from using an abstractive approach, since emails are typically comprised of a conversational nature and may not contain relevant summary-worthy text snippets.
In the same way, news summarization also features unique characteristics that affect how we summarize these texts. News articles, especially event-based journalism, usually follow an inverted pyramid scheme, in which the main facts are placed near the article's starting point. Using the first three sentences of an article as a summary is often used as a strong baseline \parencite{nenkova2005automatic}, and many systems naturally exploit position cues when extracting a summary \parencite{hong2014improving, schiffman}. For example, consider the article in Table \ref{tab:lead_ex}. While the article's first three sentences include slightly extraneous information, overall it closely mimics the reference summary and undoubtedly represents a strong extractive summary.

However, in many situations the leading sentences may not convey the most meaningful summary content. For example, consider the article in Table \ref{tab:bad_lead}. In this case, summarization models should recognize that the preamble does not contribute relevant information and exclude these sections from the output summary. Likewise, another challenge models face is recognizing that important content may occur near the end of the article, as in this example.

\begin{table}[t]
    \centering
    \small
    \begin{tabular}{|p{0.94\linewidth}|}
        \hline
        \textbf{Article}: Standing up for what you believe. What does it cost you? What do you gain? Memories Pizza in the Indiana town of Walkerton is finding out. \hl{The family-run restaurant finds itself at the center of the debate over the state's religious freedom restoration act after its owners said they'd refuse to cater a same-sex couple's wedding.} ``If a gay couple was to come and they wanted us to bring pizzas to their wedding, we'd have to say no'', Crystal O'Connor told CNN affiliate WBND-TV in South Bend. The statement struck at the heart of fears by critics, who said the new law would allow businesses to discriminate against gays and lesbians. They called for boycotts. But supporters also rallied. And by the end of the week, they had donated more than \$842,000 for the business. Social media unloaded on the pizzeria in the community of 2,100 people that few folks outside northern Indiana knew existed before this week. riskyliberal tweeted : ``Dear \#memoriespizza. no. My boycotting your business because I don't like your religious bigotry is not a violation of your freedom to practice your religion.'' ``Don't threaten \#memoriespizza'' tweeted aღanda. ``Just mock them for their ignorance.'' Bad reviews flooded the restaurant's Facebook page, most having little to do with the quality of the food. Many too vulgar to share. ``Do you really want to financially support a company that treats some of your fellow citizens like second class citizens? Boycott memories pizza!!'' said Rob Katz of Indianapolis. ``Let's hope they either rethink their policy or the free market puts them out of business.'' But \hl{one outburst in particular shut down the restaurant Wednesday and was expected to do the same Thursday.} ``Who's going to Walkerton with me to burn down Memories Pizza?'' Jessica Dooley of Goshen tweeted, according to the Walkerton police department. The account has been deleted since the tweet was posted. Detectives who investigated have recommended charges of harassment, intimidation and threats, according to Charles Kulp, assistant police chief. \\ \hline
        \textbf{Reference:} Indiana town's Memories Pizza is shut down after online threat. Its owners say they'd refuse to cater a same-sex couple's wedding. \\ \hline
    \end{tabular}
    \caption[An example in which leading sentences form a poor summary.]{In some cases, such as this one, models must learn that leading sentences do not always constitute meaningful content. As in Table \ref{tab:lead_ex}, the highlighted passages indicate sentences that reflect the content from the reference summary. The article has been truncated for conciseness.}
    \label{tab:bad_lead}
\end{table}

Previous work has shown that more than 20-30\% summary-worthy sentences come from the second half of news documents \parencite{abs3_NallapatiZSGX16, kedzie2018content}. It is therefore crucial that systems properly balance position cues with semantic representations of the text. Alas, previous studies suggest that most recent neural methods predominantly pick sentences from the lead, and that their content selection performance drops greatly when the position cues are withheld \parencite{kedzie2018content}. Table \ref{tab:lead_overlap} provides a sample of how often recent systems select sentences from the lead. We include an ``oracle'' extractive summarizer in which we compute the 3 highest-scoring sentences with respect to ROUGE, a measure of lexical overlap between the system and reference summary\footnote{See Section \ref{sec:rouge} for an overview of ROUGE.}. The striking difference between the oracle extractive summarizer and other recent systems indicates that these models' reliance on positional cues is a serious deficiency.

\begin{table}[t]
    \centering
    \begin{tabular}{|c|c|}
        \hline
        \textbf{Model} &  \textbf{Lead Overlap (\%)} \\ \hline
        Oracle & 27.24 \\ \hline
        NeuSum \parencite{neusum} & 58.24 \\ \hline
        RNES \parencite{DBLP:conf/aaai/WuH18} & 68.44 \\ \hline
        BanditSum \parencite{dong2018banditsum} & 69.87 \\ \hline
    \end{tabular}
    \caption[Frequency the lead is chosen among recent summarization models.]{On the CNN / Dailymail dataset \parencite{hermann2015teaching}, recent models frequently select leading sentences far above the rate that an oracle summarizer does.} %The oracle here corresponds to the highest-scoring triplet of sentences, according to the ROUGE metric \parencite{eva1_lin:2004:ACLsummarization}.}
    \label{tab:lead_overlap}
\end{table}

This trend is particularly worrying since it suggests that current systems ignore learning the document's details in favour of exploiting simple positional cues. This learning bottleneck has implications beyond news summarization: many other summarization domains may also hold similar idiosyncrasies that models may be exploiting instead of learning suitable representations. For this reason, it is important to design methods that promote learning beyond simple cues, and encourage models to learn deeper semantic representations.

\section{Thesis Outline}
In this thesis, we explore to what degree models are affected by positional biases in news summarization, primarily using the recent BanditSum model \parencite{dong2018banditsum} to explore these issues. We then formulate a method to counter these biases using an auxiliary target objective, showing that this technique leads to better summaries overall. We also propose a new summarization model based on classifying articles by the leading sentence's strength.
\paragraph{Chapter 2} provides the background information necessary to understand summarization. We review prior work in designing summarization systems, explaining how their development lead to the current status of the field. Particular attention is given to more recent neural extractive models, as they serve as the basis for the following experiments.
\paragraph{Chapter 3} We explore BanditSum's reliance on positional cues through a series of perturbation experiments. After showing that positional cues play a dominating role in content selection, we present a method for countering the detrimental effects of lead bias. The method estimates the value of each sentence in an article, and modifies an existing model by encouraging it to match the estimated values. We show that this technique leads to summaries that are significantly more similar to reference summaries.
\paragraph{Chapter 4} We devise a new summarization model based on classifying articles by the strength of their leading sentences. We show that by training on a subset of articles with weak leading sentences, the resulting model outperforms a baseline trained on the full dataset on this subset.
However, building models capable of performing the necessary classification remains a challenge for current methods.
\paragraph{Chapter 5} summarizes this work's main findings, and provides some potential directions for future research.

\section{Statement of Contributions}
This work was heavily influenced by colleagues' experiments, and for completeness, we include these motivating experiments here. In particular, the perturbation experiments in Chapter 3 -- Section \ref{sec:perturb} -- were conceived and executed by Yue Dong.
Original contributions from this thesis include all other experiments in Chapter 3 and all experiments in Chapter 4, i.e. the auxiliary loss and the lead classifier sections.

Parts of this thesis also appear in a paper published in the \textit{2019 Conference on Empirical Methods in Natural Language Processing and 9th International Joint Conference on Natural Language Processing} (EMNLP-IJCNLP 2019). The paper, \textit{Countering the Effects of Lead Bias in News Summarization via Multi-Stage Training and Auxiliary Losses} \parencite{grenander-etal-2019-countering}, contains all experiments from Chapters 3.

%%%%%%%%%%%%%%%%%%%%%%%%%%%%%%%%%%%%%%%%%%%%%%%%%%%%%
% Related Work
%%%%%%%%%%%%%%%%%%%%%%%%%%%%%%%%%%%%%%%%%%%%%%%%%%%%%
\chapter{Related Work}
\label{chap:relatedwork}
In this chapter, we provide an overview of select extractive summarization systems and other relevant concepts. Generally, an automatic summarization system is provided with a document $D$ and returns a summary $\mathcal{S}$ which should satisfy certain properties, such as faithfulness to the original article, non-redundancy and coherence. 

In abstractive summarization, the summary is generated token-by-token: the model typically maintains a vocabulary $V$, and at each time step $t$, it selects a word $w_t \in V$ as the next token. Abstractive summarization techniques typically must contend with large search spaces due to vocabulary sizes that are at least tens of thousands in size. Extractive summarization avoids this difficulty by creating summaries exclusively from text snippets in the source document. In the case where these snippets are sentences, an extractive summarizer can be viewed as assigning labels $y_i \in \{0, 1\}$ for each sentence $s_i$ in $D$, indicating whether $s_i$ will be included in the final summary or not.

% Multi vs. single document summarization
Summarization tasks can also differ in the number of source documents provided. In single-document summarization, a single article is summarized, whereas in multi-document summarization, the system must condense multiple, possibly overlapping sources of information. Multi-document summarization adds an extra layer of difficulty due to the high-level of redundancy across documents. In this work, we focus on the single-document setting.

Although this thesis focuses on news summarization, many datasets exist for other summarization domains. Some notable examples include legal domains such as patents and legislative documents \parencite{bigpatent, billsum}, medical documents \parencite{kedzie2018content, zhang2018radsum} and meeting notes \parencite{meeting-corpus}. Although these corpora provide exciting challenges, they are outside the scope of this work.

In this overview, we broadly cover (1) models, (2) datasets and (3) evaluation measures relevant for extractive summarization. We assume background knowledge on several basic machine learning and NLP concepts, including:
\begin{itemize}
    \item Neural networks: backpropagation, gradient descent, multi-layer perceptrons (MLP), long short-term memory networks (LSTMs), convolutional neural networks (CNNs).
    \item Natural Language Processing: ngrams, stemming, tf--idf, word embedding techniques such as Word2Vec and GloVe \parencite{word2vec, we2_pennington2014glove}.
\end{itemize}

\section{Models}
Early summarization methods, sometimes known as ``Edmundsonian'' summarizers \parencite{afantenos2005}, commonly use handcrafted features with manually tuned weights to score and rank salient sentences from the source article. Later, more sophisticated methods started borrowing methodologies from various other optimization algorithms such as PageRank and integer linear programming. As more labelled data became available, neural network-based methods such as \cite{cheng-lapata-2016-neural} began to show prominence. In addition to better summarizing performance overall, neural methods are attractive as they typically require less feature engineering.

It is important to note that the methods we present here represent a small select set of representative algorithms, and is not reflective of the entire broad field of summarization. The most relevant group of summarization systems to our approach -- neural network-based summarizers -- is presented in Section \ref{sec:neural}.

\subsection{Edmundsonian Paradigm}
% Luhn
The earliest summarization system produces abstracts for scientific papers by computing a significance factor for each sentence \parencite{luhn}. The algorithm first removes non-content words such as `is' or `and' using a lookup table. It then stems and sorts the remaining words by frequency, marking words that rank above a threshold as significant. The significance factor of a sentence is then computed as the ratio of significant words to sentence length. If the significance factor surpasses a certain threshold, it is included in the final summary.

% Edmundson
Edmundson expanded on Luhn's work with a summarization system that incorporated 4 features in its decisions \parencite{edmundson}. It considers presence of certain cue words such as `significant' and `impossible', word frequency of non-cue words, words from the article's title and heading, and word position. A weighted linear combination of the 4 features is computed for each sentence after manually determining suitable weights, and the top-ranked sentences are chosen as the summary.

\subsection{Maximum Marginal Relevance}
% MMR
Maximum Marginal Relevance (MMR) is another well-known summarization algorithm, especially for its ability to create summaries in a multi-document setting \parencite{mmr}. It measures relevance and novelty independently, then greedily selects sentences using a linear combination of these two metrics. The MMR formula is computed as:
\begin{equation}
    MMR = \argmax_{D_i \in R\setminus S} \left[\lambda Sim_1(D_i, Q) - (1 - \lambda)\max_{D_j \in S} Sim_2(D_i, D_j) \right]
\end{equation}

where $R$ is a collection of documents, $S$ is the summary so far, $D_i$ is a sentence in $R\setminus S$, $Q$ is a query vector, $\lambda$ is a hyperparameter and $Sim_1, Sim_2$ are similarity metrics, possibly identical. The algorithm greedily maximizes this formula for each sentence in the source document until the target length is achieved.

\subsection{MEAD}
% MEAD
The MEAD summarization system also focuses on the multi-document setting, but with a document-clustering approach \parencite{mead-radev}. It relies on a document-clustering algorithm named CIDR to cluster same-topic documents together. CIDR produces a vector for each cluster identified, representing the words relevant for each particular cluster. The MEAD algorithm then computes 4 features for each sentence in the source documents and scores them using a linear combination with manually-tuned weights:
\begin{equation}
    Score(S_i) = w_c C_i + w_p P_i + w_f F_i - w_R R_i
\end{equation}
where $C_i$ is a centroid value denoting the sentence's similarity to CIDR's clusters, $P_i$ is a positional value ranking earlier-occurring sentences higher, $F_i$ measures the degree of overlap with the document's first sentence and $R_S$ is a redundancy penalty term similar to MMR. Similar to MMR, sentences that maximize this score are greedily chosen until a given compression ratio is achieved.

\subsection{Graph-based Summarization}
Numerous summarization approaches have benefitted from graph-based approaches \parencite{textrank, lexrank}; we detail one of the most popular models, TextRank \parencite{textrank}. TextRank differs hugely from Edmundsonian methods, avoiding manually-tuned weights through innovative use of the PageRank algorithm. It represents a document as a graph: sentences are represented as vertices, with weighted edges between sentences determined by lexical overlap. The iterative PageRank algorithm is then applied to this graph, outputting a relevance score for each sentence. The top-ranked sentences are selected to form the final summary.

\subsection{ILP-based Summarization}
% ILP
Another family of summarization models is based around optimizing an integer linear programming problem \parencite{mcdonald2007study,clarke2008global,ilp-gillick}. We explain \cite{mcdonald2007study}'s approach, one of the earliest ILP-based summarization models. He formulates summarization as a global inference problem based on maximizing relevance while minimizing redundancy, subject to a length constraint. Given a document $\mathbf{D} = \{t_1,\dots,t_n\}$, a relevance function $Rel$, a redundancy function $Red$, and length constraint $K$, the optimal summary can be computed as:
\begin{align}
    S &= \argmax_{S \subseteq \mathbf{D}} \sum_{t_i \in S} Rel(i) - \sum_{t_i,t_j \in S,\ i < j} Red(i,j) \\
    &\text{such that} \sum_{t_i \in S} l(i) \leq K
\end{align}
where $l(i)$ is the length of $t_i$. After demonstrating that the problem is NP-hard, the author then formulates 2 approximate solutions using a greedy approach similar to MMR and dynamic programming. An exact solution can also be extracted by using an integer linear programming approach, where sentence selection is formulated as a set of linear constraints.

\subsection{Probabilistic Methods}
% SumBasic
SumBasic \parencite{sumbasic} is another well-known unsupervised summarization method. SumBasic assigns an initial probability to each word in a document based on word frequency, and ranks each sentence by the average word probability. After choosing the top-ranked sentence, SumBasic tackles redundancy issues by squaring the probabilities of each word appeared in the chosen sentence, thereby reducing its likelihood of appearing in subsequent sentence rankings:
\begin{equation}
    p_{new}(w_i) = p_{old}(w_i)\cdot p_{old}(w_i) \text{ for all } w_i \in S_i
\end{equation}
where $S_i$ is the sentence chosen at time step $i$. This process is repeated until the desired summary length is achieved.

% DPP
The last non-neural summarization method we discuss is based on determinantal point processes (DPP) \parencite{kulesza2012determinantal}. DPPs define a special type of probability measure over subsets of a fixed set of $N$ elements. An essential characteristic of DPPs is that similar elements tend not to co-occur. In other words, DPPs promote diverse subsets by assigning low probability to similar pairs of elements. This characteristic lends itself naturally to extractive summarization, where non-redundancy is an important aspect. \cite{kulesza2012determinantal} adapt DPPs to extractive summarization by computing feature vectors for each sentence in a document and training a DPP in a supervised setting. They show that DPPs are an effective method for extracting diverse, representative summaries from a document.

\subsection{Neural Summarization}\label{sec:neural}
In recent years, many researchers have shifted towards summarization methods based on deep learning, where lexical representations are learned by neural networks. Although in general these models may require more computational resources and large amounts of data to train, neural models are often considered more effective summarizers provided that the evaluation domain is similar to the training setting \parencite{ext5_summarunner}.

Neural-based extractive summarization methods typically comprise two steps. In the \textbf{sentence representation} step, models map raw text into some abstract representation, usually a vector. Then, in \textbf{sentence selection}, models use the content representations to rank and select which sentences should constitute the summary.

% Kageback
\cite{kageback-etal-2014-extractive} presented one of the earliest extractive neural summarization models, called \textbf{Continuous Vector Space Models}. The model first uses CW or Word2Vec vectors to represent words in the document \parencite{word2vec, cw}. Sentence representations are then created by summing a sentence's word embeddings or using a recursive auto-encoder (RAE) to combine word embeddings. The RAE aims to compress word embeddings recursively until single vector is left, representing the whole sentence. Sentence selection is performed by following the Lin-Bilmes method, in which a linear combination of coverage and diversity factors is approximately optimized \parencite{lin-bilmes-2011-class}.

% NN-SE
\cite{cheng-lapata-2016-neural} present a summarizer more tightly integrated with neural networks. After encoding the document's words with Word2Vec embeddings, they use a convolutional neural network (CNN) to create hidden representations $(s_1,\dots,s_m)$. In order to capture latent temporal information, these vectors are subsequently fed into a long short-term memory (LSTM) network to achieve sentence embeddings $(h_1,\dots,h_n)$. To extract sentences, another LSTM followed by a multi-layer neural network is used to predict which sentences should form the summary. The extractor sequentially labels each sentence, using the previous time step's decision to help inform whether or not to include the current sentence in the final summary:
\begin{align}
    \Bar{h}_t &= \mathrm{LSTM}(p_{t-1}s_{t-1}, \Bar{h}_{t-1})\\
    p(y_t = 1 | D) &= \sigma(\mathrm{MLP}([\Bar{h}_{t}; h_t]))
\end{align}
where $[\Bar{h}_{t}; h_t]$ denotes concatenation of the two vectors.

Large neural models ordinarily require large amounts of training data, which was not readily available at the time. To overcome this data paucity, the authors retrieved hundreds of thousands of news articles from the Daily Mail news archives, along with associated bullet point highlights to serve as summaries. The model is then trained using this resource.

% SummaRuNNer
\cite{ext5_summarunner}'s \textbf{SummaRuNNer} model employs two layers of bidirectional GRUs to create sentence representations. The first biGRU runs over the document's words to create word-level representations. Each sentence's word-level representations are then averaged and fed into another GRU to create sentence-level representations $(h_1,\dots,h_n)$. A document representation $\mathbf{d}$ is also computed by averaging sentence embeddings followed by a non-linear transformation.

A logistic layer then classifies whether to include each sentence or not based on a variety of factors:
\begin{equation}\label{eq:autoregressive-1}
    p(y_j = 1 | h_j, s_j, \mathbf{d}) = \sigma\left(W_c h_j + h_j^T W_s \mathbf{d} - h_j^T W_r \tanh{(s_j)} + W_{ap} p_j^a + W_{rp}p_j^r + b \right)
\end{equation}
where the $W_x$ and $b$ variables represent learnable parameters.
The ``summary-so-far'' representation $s_j$ is recursively computed by weighting each sentence representation by the probability outputted by the logistic classifier: 
\begin{equation}\label{eq:autoregressive-2}
    s_j = \sum_{i=1}^{j-1} h_i p(y_i = 1 | h_i, s_i, d)
\end{equation}
Finally, positional embeddings $p_j^a$ and $p_j^r$ indicating the absolute and relative sentence position are included as inputs to the logistic classifier.

% Refresh
\cite{DBLP:Narayan/2018} argue that minimizing a cross-entropy loss objective, as done in many previous neural models, is not ideal for supervised summarization tasks. In order to use cross-entropy loss for supervised summarization tasks, it is often necessarily to create heuristic gold labels. \cite{DBLP:Narayan/2018} note that mismatches exist between heuristic labels and true summary-worthy content, which may hurt summarization performance. Instead, they argue that directly optimizing ROUGE can prevent these errors. In order to optimize ROUGE, the authors formulate a novel objective based on the REINFORCE algorithm \parencite{williams1992simple}. Their model architecture itself is similar to \cite{cheng-lapata-2016-neural}: words are first encoded by CNN followed by a LSTM to create sentence embeddings. An LSTM followed by a softmax layer then assigns a probability score to each sentence. During training, sentences with low ROUGE scores are manually filtered to reduce the search space, and a summary is created by sampling from the remaining probabilities. After creating a summary hypothesis, the model is scored against the reference summary using ROUGE, and this score is backpropagated through the network using REINFORCE.

% BanditSum
\textbf{BanditSum} is another recent neural model trained using a reinforcement learning-based objective function. Unlike \cite{DBLP:Narayan/2018}, BanditSum does not filter the action space, and thereby samples from the true action space instead of approximating it. BanditSum employs two sets of bidirectional LSTMs to create sentence representations. After embedding words with GloVe \parencite{we2_pennington2014glove}, a word-level LSTM creates word representations of the article. For each sentence, the word-level representations are averaged and inputted into a sentence-level LSTM to create sentence representations $(h_1,\dots,h_n)$. A multi-layer perceptron then runs over the sentence representations to create \textit{sentence affinity} scores for each sentence. Similar to \cite{DBLP:Narayan/2018}, hypothesis summaries are created by sampling without replacement from the sentence affinities. After sampling $B$ distinct summaries, the extracts are scored using ROUGE and the weights are updated using the REINFORCE algorithm \parencite{williams1992simple}.

\subsubsection{Language Model Pre-Training}
Recently, summarization methods based on language model pre-training objectives have achieved state-of-the-art results. These approaches, released concurrently with our work, are particularly interesting because they tend to exhibit less lead bias, and thus provide promising directions for future research. We detail here the extractive model BertSum \parencite{bertsum}, though abstractive approaches based on large pre-training objectives also exist \parencite{lewis2019bart}.

BertSum is an extractive model that uses BERT to encode sentences \parencite{bertsum}. BERT is a language representation model built with a Transformer architecture \parencite{vaswani2017attentionTransformer} and trained with a masked language modeling task. It has proven effective as an encoder for both words and sentences in many NLP tasks. \cite{bertsum} use the pre-trained BERT model to encode sentences in a document, followed by a Transformer-based decoder to rank sentences by summary-worthiness. The model is fine-tuned with a cross-entropy loss using heuristically-generated gold labels.

\section{Summarization Evaluation}
\subsection{Human Evaluation}
Traditionally, evaluating summary quality has been performed by human judges. Reference and systems summaries are compared along multiple dimensions, such as coverage, non-redundancy and coherence. % Cite some 
However, although human summary evaluation is usually considered to be the most conclusive method to compare summarization systems, it is often difficult to scale human summary evaluation to tens of thousands of summary judgements. Moreover, human judgement has not been standardized in the summarization community, and different authors often use incompatible metrics when evaluating summary quality. These inconsistencies complicate comparisons between different systems.

\subsection{ROUGE}\label{sec:rouge}
Driven by the difficulties of human evaluation, automatic summary evaluation has been researched for many years. The most common evaluation scheme is ROUGE \parencite{eva1_lin:2004:ACLsummarization}, a set of metrics based on measuring word overlap between the generated system summary and a reference summary. ROUGE is comprised of several individual measures, representing distinct aspects of comparison. In our experiments, we report the following metrics:
\begin{itemize}
    \item ROUGE-N measures the ngram overlap between the generated and reference summaries. Given a set of reference summaries $\mathcal{R} = \{\mathcal{R}_1,\dots,\mathcal{R}_m\}$ and a system summary $\mathcal{S}$, the ROUGE-N score is computed as:
    \begin{equation}
        \rougen(\mathcal{S}, \mathcal{R}) = \frac{\sum_{i=1}^m  \sum_{gram_N \in \mathcal{R}_i} Count_{match(\mathcal{S}, \mathcal{R}_i)}(gram_N)}{\sum_{i=1}^m \sum_{gram_N \in \mathcal{R}_i} Count(gram_N)}
    \end{equation}
    where $N$ is the ngram length we are considering, $gram_N$ is an ngram of length $N$, and $Count_{match(\mathcal{S}, \mathcal{R}_i)}$ is the number of matching ngrams between $\mathcal{R}_i$ and $\mathcal{S}$.
    
    Although the original paper describes this metric as a recall-based measure, recent models have at times reported the $\mathrm{F_1}$ score instead. This decision is usually based on the dataset and whether a summary length limit is specified. To better compare with state-of-the-art models, we compute the F1 score. In our experiments, we report both ROUGE-1 (unigram) and ROUGE-2 (bigram) metrics.
    \item ROUGE-L scores how well the word order is respected in the reference summary. It is based on a ``\textit{union} longest common subsequence'' score $LCS_{\cup}$ between the system summary and each sentence in the reference summary \parencite{eva1_lin:2004:ACLsummarization}. 
    Given a system summary $\mathcal{S}$ with sentences $s_1,\dots,s_v$ and a reference summary $\mathcal{R}$ with sentences $r_1,\dots,r_u$, we first create sets from the longest common subsequence (LCS) between each reference sentence $r_i$ and $\mathcal{S}$.
    $LCS_{\cup}(r_i, \mathcal{S})$ is computed by taking the union of these sets and dividing by the total word length of $r_i$:
    \begin{equation}
        LCS_{\cup}(r_i, \mathcal{S}) = \frac{\vert \bigcup_{i=1}^u LCS(r_i, \mathcal{S}) \vert}{\vert r_i \vert}
    \end{equation}
    The recall and precision scores are then computed as:
    \begin{equation}
        R_{lcs} = \frac{\sum_{i=1}^u LCS_{\cup}(r_i, \mathcal{S})}{\vert\mathcal{R}\vert} \quad\text{, }\quad
        P_{lcs} = \frac{\sum_{i=1}^u LCS_{\cup}(r_i, \mathcal{S})}{\vert\mathcal{S}\vert} 
    \end{equation}
    The final ROUGE-L F1 score is given by the F1 score of the above precision and recall.
\end{itemize}

\section{Datasets}
% DUC and TAC
The Document Understanding Conferences\footnote{\url{https://duc.nist.gov/}} (DUC) produced an annual summarization dataset from 2001--2007, before joining as a track within the Text Analysis Conferences\footnote{\url{https://tac.nist.gov/}} (TAC) \parencite{harman-over-2004-effects, dang-2006-duc, dang2008overview}. The DUC/TAC datasets consist of news articles paired with multiple human-written summaries, making these datasets an excellent resource for multi-document summarization. However, the datasets generally do not contain enough samples to train neural network-based summarization models.

% CNN / Daily Mail
The CNN / Daily Mail dataset is a large collection of news stories with associated bullet point highlights used as summaries \parencite{hermann2015teaching}. The dataset is split into 287,227/13,368/11,490 article-summary pairs for the training / development / testing datasets, making it a tremendous resource for neural methods which often require large quantities of training data for adequate generalization. The CNN / Daily Mail is especially suitable for extractive systems as the summaries are known to be quite extractive in nature -- in other words, summary content is often copied from the source document  \parencite{grusky2018newsroom}. 

While other news summarization datasets exist, such as the New York Times Corpus, Newsroom and XSum \parencite{nyt-corpus, grusky2018newsroom, xsum}, they are more abstractive and hence less suitable for our purposes. 
Gigaword is another well-known summarization dataset containing millions of news articles from several publishers \parencite{napoles-etal-2012-annotated}; however, the news articles are not paired with summaries, complicating the use of machine learning approaches.
In all experiments, we report results using the CNN / Daily Mail dataset.

%%%%%%%%%%%%%%%%%%%%%%%%%%%%%%%%%%%%%%%%%%%%%%%%%%%%%
% Conclusion
%%%%%%%%%%%%%%%%%%%%%%%%%%%%%%%%%%%%%%%%%%%%%%%%%%%%%
\chapter{Conclusion}
In this work, we have investigated summarization systems and how positional biases affect these systems' learning process. Motivated by \cite{kedzie2018content}'s work and our own experiments showing that current summarization systems are heavily affected by positional cues, we design novel methods to counter the dominance of these signals.

Our first method involves augmenting an existing gradient descent-based summarization model with an auxiliary loss objective. For a given input article, this loss objective estimates each sentence's value by computing the sentence-level ROUGE score compared to the reference summary. It then encourages the model to match these sentence-value estimates using the KL divergence between the model's predictions and the estimates. We test this method with a state-of-the-art summarization system, BanditSum \parencite{dong2018banditsum}, and find that our method can significantly improve summary quality. 

Although the improvements are noteworthy, the resulting model is still hampered by an overreliance on lead bias. Evidence for this lingering issue is seen in Figure \ref{fig:avg_pos}, as the oracle summarizer extracts leading sentences far less than the BanditSum+KL method. Some extensions to this method could include other algorithms to combine the loss functions such as MAML \parencite{maml}, or investigating RL methods aimed at promoting exploration such as the Soft-Actor Critic algorithm \parencite{sac}.

After hypothesizing that articles with a strong vs. weak lead require different summarization strategies, we design a second novel summarization method based around classifying whether an article's lead contains summary-worthy sentences. We find that by partitioning the training dataset and training separate summarization systems on these subsets, we can achieve greater performance on both subsets. We note that in the case of the \Dearly{} subset, the model's learning is dominated by positional bias, and it learns to mimic the lead-3 baseline exactly. Training a classifier proves to be a much more difficult task, especially with regards to overfitting.

% Limitations
Assuming an accurate classifier is achievable, future summarization approaches may render this approach redundant. Emerging summarization models, such as BertSum and BART \parencite{bertsum, lewis2019bart}, have taken advantage of large unsupervised language model pre-training in their approaches. In particular, the authors of BertSum show that their method is able extract sentences further in the document compared to a competitive baseline, and that the extracted indices are more similar to an oracle summarizer. Compared to previous models such as BanditSum, BertSum is more robust with respect to lead bias effects, despite having no explicit target objective aimed at countering this damaging signal. This may mean that explicit lead bias regularization is not necessary, as long as the model can sufficiently balance positional cues with semantic ones.

% Conclude: lead bias is an important factor for summarization
Regardless of how future summarization models operate, we have shown that positional biases are an important consideration when designing summarization systems. Creating models that understand how to properly balance positional cues and value sentences correctly represents a major milestone towards truly practical summarization systems.


%%%%%%%%%%%%%%%%%%%%%%%%%%%%%%%%%%%%%%%%%%%%%%%%%%%%%
% Bibliography
%%%%%%%%%%%%%%%%%%%%%%%%%%%%%%%%%%%%%%%%%%%%%%%%%%%%%
%\bibliography{library}
%\bibliographystyle{plain}
\printbibliography
\end{document}