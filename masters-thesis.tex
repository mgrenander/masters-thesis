% Packages (fold)
\RequirePackage{lmodern}
\documentclass[12pt, oneside, extrafontsizes]{memoir}  % TODO 12pt, twoside

\setstocksize{11in}{8.5in}
\settrimmedsize{11in}{8.5in}{*}
\settrims{0in}{0in}
\setlrmarginsandblock{25mm}{25mm}{*}
\setulmarginsandblock{25mm}{25mm}{*}
\setheadfoot{13pt}{26pt}
\setheaderspaces{*}{13pt}{*}
\checkandfixthelayout
\DoubleSpacing
\setsecnumdepth{subsubsection}
\headstyles{default}
\chapterstyle{ell}
\setsecheadstyle{\scshape\LARGE\raggedright}

\usepackage[colorlinks,bookmarksnumbered,bookmarksdepth=subsubsection,unicode=true]{hyperref}
\newsubfloat{figure}  % must follow hyperref
\hypersetup{
pdfauthor = {Matt Grenander},
pdftitle = {Learning to Balance Lead Bias in News Summarization},
pdfsubject = {Subject},
pdfkeywords = {natural language processing, automatic summarization, deep learning, machine learning},
pdfcreator = {LaTeX with the hyperref package},
pdfproducer = {},
linkcolor = [HTML]{000000},
citecolor = [HTML]{0000FF},
%urlcolor = [HTML]{\colorc}
}
\usepackage{amsmath}
\usepackage{amsfonts}
\usepackage{amssymb}
\usepackage{amsthm}
\usepackage{dsfont}
\usepackage{xspace}
\usepackage{tikz}
\usetikzlibrary{shapes,arrows}
\usepackage{standalone}
\usepackage{color,soul}
\usepackage[boxed]{algorithm2e}

% Custom commands
\newcommand{\rvs}{r.v.'s\xspace}
\newcommand{\mdp}{Markov Decision Process\xspace}
\newcommand{\mdps}{Markov Decision Processes\xspace}
\newcommand{\mrp}{Markov Reward Process\xspace}
\newcommand{\mc}{Markov Chain\xspace}

\def\given{\;\middle\vert\;}
\def\optimal{\star}
\def\transpose{\intercal}
\def\laplacian{\mathbf{\mathcal{L}}}
\def\eqdef{\overset{\underset{\mathrm{def}}{}}{=}}
\def\indicator{\mathds{1}}
\DeclareMathOperator{\expectation}{\mathbb{E}}
\DeclareMathOperator{\vol}{\text{vol}}

\newcommand{\Drandom}{$D_{\mathrm{random}}$}
\newcommand{\Dinorder}{$D_{\mathrm{inorder}}$}
\newcommand{\Dearly}{$D_{\mathrm{early}}$}
\newcommand{\Dlate}{$D_{\mathrm{late}}$}
\newcommand{\Dmedian}{$D_{\mathrm{med}}$}

\newcommand{\todo}[1]{[TODO: #1]}
\newcommand{\termidx}[1]{\index{#1}{\textbf{#1}}}

\theoremstyle{plain}
\newtheorem{thm}{Theorem}[section]
\newtheorem{lem}[thm]{Lemma}
\newtheorem{prop}[thm]{Proposition}
\newtheorem*{cor}{Corollary}

\theoremstyle{definition}
\newtheorem{defn}{Definition}[section]
\newtheorem{conj}{Conjecture}[section]
\newtheorem{exmp}{Example}[section]

\usepackage[utf8]{inputenc}
\usepackage{csquotes}
\usepackage{showidx}
\makeindex

\usepackage[backend=biber, citestyle=authoryear, bibstyle=authoryear, isbn=false, url=false, doi=false, eprint=false, natbib=false, sorting=nty, uniquename=init]{biblatex}
\addbibresource{library.bib}

\begin{document}

%%%%%%%%%%%%%%%%%%%%%%%%%%%%%%%%%%%%%%%%%%%%%%%%%%%%%
% Title page
%%%%%%%%%%%%%%%%%%%%%%%%%%%%%%%%%%%%%%%%%%%%%%%%%%%%%

% Title (fold)
\pretitle{\begin{center}\cftchapterfont\huge}
\posttitle{\end{center}}
\preauthor{\begin{center}\huge}
\postauthor{\end{center}}
\predate{\begin{center}\large}
\postdate{\end{center}}

\title{Learning to Balance Lead Bias in News Summarization}
\author{Matt Grenander}
\date{\today}
\renewcommand\maketitlehookb{
\vfill
}
\renewcommand\maketitlehookc{
\vfill
\begin{center}
{
\large
Department of Computer Science\\
McGill University, Montreal
}
\end{center}
\vspace{10mm}
}
\renewcommand\maketitlehookd{
\vspace{10mm}
\begin{center}
A thesis submitted to McGill University in partial fulfilment of the requirements of
the degree of Master of Science. \\
\copyright 2020 Matt Grenander
\end{center}
}
% Title (end)

\begin{titlingpage}
\maketitle
\end{titlingpage}

%%%%%%%%%%%%%%%%%%%%%%%%%%%%%%%%%%%%%%%%%%%%%%%%%%%%%
% Ackowledgements
%%%%%%%%%%%%%%%%%%%%%%%%%%%%%%%%%%%%%%%%%%%%%%%%%%%%%
%\clearpage
%\pagenumbering{roman}
%\renewcommand{\abstractname}{Dedication}

%%%%%%%%%%%%%%%%%%%%%%%%%%%%%%%%%%%%%%%%%%%%%%%%%%%%%
% Ackowledgements
%%%%%%%%%%%%%%%%%%%%%%%%%%%%%%%%%%%%%%%%%%%%%%%%%%%%%
\clearpage
\pagenumbering{roman}
\renewcommand{\abstractname}{Acknowledgements}
\begin{abstract}
Acknowledgements go here.
\end{abstract}

%%%%%%%%%%%%%%%%%%%%%%%%%%%%%%%%%%%%%%%%%%%%%%%%%%%%%
% Abstract
%%%%%%%%%%%%%%%%%%%%%%%%%%%%%%%%%%%%%%%%%%%%%%%%%%%%%
\clearpage
\renewcommand{\abstractname}{Abstract}
\begin{abstract}
The progression of technology and the Internet has allowed us to create and share information at an unprecedented rate.
The usefulness of summarization systems to intelligently condense text has become apparent as a method to sift through the vast amount of text we encounter each day.
Summarization systems typically follow one of two approaches.
In abstractive summarization, the system generates a summary token-by-token, whereas in the extractive approach, the system is tasked with selecting snippets of text from the source document -- typically sentences -- that best represent the text.
Extractive summarization systems have recently been shown to exhibit a bias towards selecting content from an article's lead sentences, even when these sentences are completely irrelevant to the overall text. \parencite{kedzie2018content}.
We investigate this phenomenon in detail, showing that when articles' sentence order is permuted, state-of-the-art systems cannot recover the same performance.
We then propose a solution to this problem, an auxiliary objective function which encourages the model to look beyond a document's leading sentences and properly value each sentence.
We show that adding this auxiliary objective results in significant improvements in performance, particularly in cases where the article's leading sentences constitute a poor summary.
We extend this approach to a novel summarization method that classifies documents with a strong lead vs. a weak lead, and having separate systems summarize the two cases.
We show that this approach is promising, though more work is needed in classifying articles in this way accurately.
\end{abstract}

\clearpage
\renewcommand{\abstractname}{Abrégé}
\begin{abstract}
French translation goes here.
\end{abstract}

%%%%%%%%%%%%%%%%%%%%%%%%%%%%%%%%%%%%%%%%%%%%%%%%%%%%%
% Table of content
%%%%%%%%%%%%%%%%%%%%%%%%%%%%%%%%%%%%%%%%%%%%%%%%%%%%%
\clearpage
\setcounter{tocdepth}{2}
\tableofcontents
\newpage
\listoffigures
%\listofalgorithms

%%%%%%%%%%%%%%%%%%%%%%%%%%%%%%%%%%%%%%%%%%%%%%%%%%%%%
% Introduction
%%%%%%%%%%%%%%%%%%%%%%%%%%%%%%%%%%%%%%%%%%%%%%%%%%%%%
\clearpage
\pagenumbering{arabic}
\chapter{Introduction}
Extractive summarization remains a simple 
and fast approach to produce summaries which 
are grammatical and accurately represent the source text.
In the news domain, these systems are able to 
use a dominant signal: the position of a sentence 
in the source document. 
Due to journalistic conventions which place important information
early in the articles, the lead sentences often contain key information. In this paper, we explore how systems can look beyond this simple trend. 

Naturally, automatic systems have 
all along exploited position cues in news 
as key indicators of important content \parencite{schiffman,hong2014improving,ext_bert}. 
The `lead' baseline is rather strong in single-document news summarization \parencite{brandow1995automatic,nenkova2005automatic}, 
with automatic systems only modestly improving the results. 
Nevertheless, more than 20-30\% of 
summary-worthy sentences come from the second half of news documents \parencite{data2_nallapati2016abstractive,kedzie2018content}, 
and the lead baseline, as shown in Table \ref{tab:lead_ex},
does not always produce convincing summaries.
So, systems must balance the position bias with representations of the semantic content 
throughout the document. Alas, preliminary studies
\parencite{kedzie2018content} suggest that even the most recent neural 
methods predominantly pick sentences from the lead, and 
that their content selection performance drops greatly
when the position cues are withheld.

\begin{table}[t]
    \centering
    \small
    \begin{tabular}{|p{0.94\linewidth}|}
        \hline
        \textbf{Lead-3:} Bangladesh beat fellow World Cup quarter-finalists Pakistan by 79 runs in the first one-day international in Dhaka. Tamim Iqbal and Mushfiqur Rahim scored centuries as Bangladesh made 329 for six and Pakistan could only muster 250 in reply. Pakistan will have the chance to level the three-match series on Sunday when the second odi takes place in Mirpur. \\ \hline
        \textbf{Reference:} Bangladesh beat fellow World Cup quarter-finalists Pakistan by 79 runs. Tamim Iqbal and Mushfiqur Rahim scored centuries for Bangladesh. Bangladesh made 329 for six and Pakistan could only muster 250 in reply. Pakistan will have the chance to level the three-match series on Sunday. \\ \hline \hline
        \textbf{Lead-3}: Standing up for what you believe. What does it cost you? What do you gain? \\ \hline
        \textbf{Reference:} Indiana town's Memories Pizza is shut down after online threat. Its owners say they'd refuse to cater a same-sex couple's wedding. \\
        \hline
    \end{tabular}
    \caption{`Lead' (first 3 sentences of source) can produce extremely faithful (top) to disastrously inaccurate (bottom) summaries. Gold standard summaries are also shown.}
    \label{tab:lead_ex}
\end{table}

In this paper, we verify that 
sentence position and lead bias dominate the 
learning signal for  state-of-the-art neural 
extractive summarizers in the news domain. 
We then present techniques to improve content selection in 
the face of this bias. 
The first technique makes use of `unbiased data' 
created by permuting the order of sentences in the training
articles. We use this shuffled dataset for pre-training, followed by 
training on the original (unshuffled) articles. 
The second method introduces an auxiliary loss 
which encourages the model's scores for sentences 
to mimic an estimated score distribution over the
sentences, the latter computed using ROUGE overlap 
with the gold standard. We implement these techniques 
for two recent reinforcement learning based systems, 
RNES \parencite{DBLP:conf/aaai/WuH18} and 
BanditSum \parencite{dong2018banditsum}, and evaluate 
them on the CNN/Daily Mail dataset \parencite{hermann2015teaching}.

We find that our auxiliary loss achieves 
significantly better ROUGE scores
compared to the base systems, and that the 
improvement is even more pronounced when the true 
best sentences appear later in the article. 
On the other hand, the pretraining approach produces mixed results.
We also confirm that when summary-worthy sentences 
appear late, there is a large performance discrepancy 
between the oracle summary and state-of-the-art summarizers,
indicating that learning to balance lead bias with 
other features of news text is a noteworthy issue to tackle.

\subsection{Outline}

Basic theory of stochastic processes and Markov chains is first presented in chapter \ref{chap:decisionmaking}. The inclusion of this material is motivated by the probabilistic interpretation of spectral graph theory studied in chapter \ref{chap:dynamics}. The theory of Markov Decision Processes is presented in section \ref{sec:mdp} as a necessary prerequisite for the appreciation of the techniques developed in reinforcement learning. Chapter \ref{chap:temporalabstraction} focuses on the presentation of the options framework \parencite{Sutton1999} in reinforcement learning as a way to represent temporal abstraction and learn optimal control over it.

The connection from graph \textit{structure} to system \textit{dynamics} is developed throughout chapter \ref{chap:dynamics} and is instrumental is understanding the strengths and pitfalls of the graph partitioning approach for options discovery. It also allows a better understanding of the relevant work on Nearly-Completely Decomposable Markov Chains (NCD) for future theoretical research on the bottleneck concept. The NCD theory seems to call for an information theoretic comprehension of temporal abstraction which is briefly developed at the end of this section.

A new algorithm for options discovery is proposed in chapter \ref{chap:buildingoptions} based on the  \textsc{Walktrap} community detection algorithm of \cite{Pons2005}. Although \textsc{Walktrap} finds its roots into spectral graph theory, its running time is only order $\mathcal{O}(mn^2)$ rather than $\mathcal{O}(n^3)$ by avoiding to compute the eigenvectors explicitly. The problem of options discovery and construction is also set under the assumption of a continuous state space. Techniques for constructing proximity graphs in Euclidean space are developed in section \ref{sec:proximitygraphs}. Section \ref{sec:knnoptions} shows how approximate nearest neighbors algorithms can be used to properly define the initiation and termination components of options under continuous observations.

An illustration of the proposed algorithm is provided in chapter \ref{chap:illustration} with the Pinball domain of \cite{Konidaris2009}. Practical difficulties having to do oscillations and off-policy learning are analysed. The proper empirical choices for the number of nearest neighbors, type of proximity graph and time scale for the \textsc{Walktrap} algorithm are discussed.

%%%%%%%%%%%%%%%%%%%%%%%%%%%%%%%%%%%%%%%%%%%%%%%%%%%%%
% Related Work
%%%%%%%%%%%%%%%%%%%%%%%%%%%%%%%%%%%%%%%%%%%%%%%%%%%%%
\chapter{Related Work}
\label{chap:relatedwork}

Modern summarization methods for news are typically
based on neural network-based sequence-to-sequence learning
\parencite{cnn1_kalchbrenner2014convolutional,cnn2_kim2014convolutional,rnn2_chung2014gru,ext2_2015Yin,ext3_cao2015learning,ext4_cheng2016neural,ext5_summarunner,narayan2018don,neusum}.
In MLE-based training, extractive summarizers are 
trained with gradient ascent to maximize the likelihood
of heuristically-generated ground-truth binary
labels \parencite{ext5_summarunner}. Many MLE-based models
do not perform as well as their reinforcement 
learning-based (RL) competitors that directly optimize ROUGE \parencite{abs5_paulus2017deep,DBLP:Narayan/2018,dong2018banditsum,DBLP:conf/aaai/WuH18}. 
As RL-based models represent the state of the art for 
extractive summarization, we analyze them in this paper.

The closest work to ours is a recent study by 
\cite{kedzie2018content} which showed that 
MLE-based models learn a significant bias for 
selecting early sentences when trained on news 
articles as opposed to other domains. As much as 
58\% of selected summary sentences come directly 
from the lead. Moreover, when these models
are trained on articles whose sentences
are randomly shuffled, the performance drops 
considerably for news domain only. While this drop 
could be due to the destruction of position cues, 
it may also arise because the article's coherence
and context were lost. 

In this paper, we employ finer control on the 
distortion of sentence position, coherence, and 
context, and confirm that performance drops are 
mainly due to the lack of position cues. 
We also propose the first techniques to 
counter the effects of lead bias in neural extractive systems.


%%%%%%%%%%%%%%%%%%%%%%%%%%%%%%%%%%%%%%%%%%%%%%%%%%%%%
% Conclusion
%%%%%%%%%%%%%%%%%%%%%%%%%%%%%%%%%%%%%%%%%%%%%%%%%%%%%
\chapter{Conclusion}
In this work, we have investigated summarization systems and how positional biases affect these systems' learning process. Motivated by \cite{kedzie2018content}'s work and our own experiments showing that current summarization systems are heavily affected by positional cues, we design novel methods to counter the dominance of these signals.

Our first method involves augmenting an existing gradient descent-based summarization model with an auxiliary loss objective. For a given input article, this loss objective estimates each sentence's value by computing the sentence-level ROUGE score compared to the reference summary. It then encourages the model to match these sentence-value estimates using the KL divergence between the model's predictions and the estimates. We test this method with a state-of-the-art summarization system, BanditSum \parencite{dong2018banditsum}, and find that our method can significantly improve summary quality. 

Although the improvements are noteworthy, the resulting model is still hampered by an overreliance on lead bias. Evidence for this lingering issue is seen in Figure \ref{fig:avg_pos}, as the oracle summarizer extracts leading sentences far less than the BanditSum+KL method. Some extensions to this method could include other algorithms to combine the loss functions such as MAML \parencite{maml}, or investigating RL methods aimed at promoting exploration such as the Soft-Actor Critic algorithm \parencite{sac}.

After hypothesizing that articles with a strong vs. weak lead require different summarization strategies, we design a second novel summarization method based around classifying whether an article's lead contains summary-worthy sentences. We find that by partitioning the training dataset and training separate summarization systems on these subsets, we can achieve greater performance on both subsets. We note that in the case of the \Dearly{} subset, the model's learning is dominated by positional bias, and it learns to mimic the lead-3 baseline exactly. Training a classifier proves to be a much more difficult task, especially with regards to overfitting.

% Limitations
Assuming an accurate classifier is achievable, future summarization approaches may render this approach redundant. Emerging summarization models, such as BertSum and BART \parencite{bertsum, lewis2019bart}, have taken advantage of large unsupervised language model pre-training in their approaches. In particular, the authors of BertSum show that their method is able extract sentences further in the document compared to a competitive baseline, and that the extracted indices are more similar to an oracle summarizer. Compared to previous models such as BanditSum, BertSum is more robust with respect to lead bias effects, despite having no explicit target objective aimed at countering this damaging signal. This may mean that explicit lead bias regularization is not necessary, as long as the model can sufficiently balance positional cues with semantic ones.

% Conclude: lead bias is an important factor for summarization
Regardless of how future summarization models operate, we have shown that positional biases are an important consideration when designing summarization systems. Creating models that understand how to properly balance positional cues and value sentences correctly represents a major milestone towards truly practical summarization systems.


%%%%%%%%%%%%%%%%%%%%%%%%%%%%%%%%%%%%%%%%%%%%%%%%%%%%%
% Bibliography
%%%%%%%%%%%%%%%%%%%%%%%%%%%%%%%%%%%%%%%%%%%%%%%%%%%%%
%\bibliography{library}
%\bibliographystyle{plain}
\printbibliography
\end{document}